\chapter{ANALISIS DAN PERANCANGAN SISTEM}


%------------------------------------------------------------------------------
% Analisis Sistem
%------------------------------------------------------------------------------
\section{Analisis Sistem}
Pada penelitian ini dibuat sistem untuk mengenali aktivitas manusia secara \textit{real time} dengan sensor akselerometer dan giroskop yang tertanam pada ponsel cerdas. Rancangan sistem ini dapat dilihat pada Gambar~\ref{gambar:diagram-blok-sistem}.

\begin{figure}[h!]
    \centering
    \begin{tikzpicture}[node distance=1.5cm]
        \node (dataset) [process] {Dataset};
        \node (pelatihan) [process, right of=dataset, xshift=2cm] {Pelatihan};
        \node (model-klasifikasi) [process, right of=pelatihan, xshift=5cm, minimum width=5cm] {Model Klasifikasi};  
        \node (akselerometer) [process, above of=model-klasifikasi, xshift=-6.25cm, yshift=1cm] {Akselerometer};
        \node (giroskop) [process, right of=akselerometer, xshift=2cm] {Giroskop};
        \draw (2,1.5) -- (9,1.5) -- (9,4) -- node[anchor=north] {Sensor} (2,4) -- (2,1.5);
        \node (aktivitas) [process, below of=model-klasifikasi, yshift=-0.5cm] {Aktivitas};

        \draw [arrow] (dataset) -- (pelatihan);
        \draw [arrow] (pelatihan) -- (model-klasifikasi);
        \draw [arrow] (9,2.5) -- (10,2.5) -- (model-klasifikasi);
        \draw [arrow] (2,2.5) -- (0,2.5) -- (dataset);
        \draw [arrow] (model-klasifikasi) -- node[anchor=west, font=\footnotesize] {Pengenalan} (aktivitas);
        \draw [dashed] (6.25, 1) -- (6.25, -2.5);
    \end{tikzpicture}
    \caption{Diagram blok sistem}
    \label{gambar:diagram-blok-sistem}
\end{figure}

Sistem ini memanfaatkan model klasifikasi yang disusun dari lapisan-lapisan \textit{Convolutional Neural Network} (CNN) dan \textit{Long Short-Term Memory} (LSTM) untuk mengklasifikasikan bacaan sensor akselerometer dan giroskop menjadi enam aktivitas sederhana, yaitu duduk, berdiri, berjalan, berlari, menaiki tangga dan menuruni tangga.

Sebelum dapat melakukan klasifikasi, model dilatih dengan dataset aktivitas manusia. Dalam proses pelatihan tersebut, model klasifikasi menerima masukan dari data latih lalu mengoptimasi parameter-parameternya untuk menghasilkan klasifikasi yang lebih baik. Setelah dilatih, model diimplementasikan pada ponsel cerdas Android untuk melakukan pengenalan aktivitas manusia secara \textit{real time} dari data sensor akselerometer dan giroskop yang tertanam.

%------------------------------------------------------------------------------
% Peralatan
%------------------------------------------------------------------------------
\section{Peralatan}
Peralatan yang digunakan untuk mendukung penelitian ini terdiri dari perangkat keras dan perangkat lunak berikut:

\subsection{Perangkat Keras}
\begin{enumerate}
    \item Laptop HP 14--015tx dengan CPU Intel Core i5--6200U 2.3 GHz dan RAM 12GB
    \item Ponsel cerdas Android dengan sensor akselerometer dan giroskop tertanam
\end{enumerate}

\subsection{Sistem Perangkat Lunak}
\begin{enumerate}
    \item Ubuntu 16.04 64 bit sebagai sistem operasi laptop
    \item TensorFlow sebagai pustaka komputasi numerik
    \item Visual Studio Code sebagai \textit{text editor} untuk menulis program sistem klasifikasi
    \item Android Studio sebagai \textit{Integrated Development Environment} (IDE) untuk mengembangkan aplikasi Android
\end{enumerate}

%------------------------------------------------------------------------------
% Rancangan Perangkat Lunak
%
% TODO:
% - Offline:
%   - Model klasifikasi
%   - Pelatihan
%   - Test
% - Online:
%   - Klasifikasi
%
%------------------------------------------------------------------------------
\subsection{Rancangan Perangkat Lunak}
Sistem dirancang untuk mengklasifikasikan aktivitas manusia berdasarkan data dari sensor akselerometer dan giroskop pada ponsel cerdas. Aliran data sensor pada proses klasifikasi dan pelatihan ditunjukkan pada Gambar~\ref{gambar:aliran-data-klasifikasi}. Diagram tersebut menggambarkan graf komputasi model klasifikasi serta proses pelatihan yang dilakukan untuk mengoptimasi parameter-paramter pada model.

\begin{figure}[h!]
    \centering
    \begin{tikzpicture}[node distance=1.9cm]
        \node (input) [state] {Input};
        \node (cnn) [state, above of=input, yshift=0.5cm] {Konvolusi};
        \node (lstm) [state, above of=cnn, yshift=0.8cm] {LSTM};
        \node (dropout) [state, above of=lstm, yshift=0.6cm] {Dropout};
        \node (softmax) [state, above of=dropout, yshift=0.6cm] {Softmax};
        \node (output) [state, above of=softmax, yshift=0.5cm] {Output};
        \node (loss) [state, right of=softmax, xshift=3cm] {Loss};
        \node (akurasi) [state, right of=output, , xshift=3cm] {Akurasi};
        \node (target) [state, right of=loss, yshift=1.2cm] {Target};
        \node (optimizer) [state, right of=lstm, xshift=3cm] {Optimizer};

        \draw [arrow] (input) -- (cnn);
        \draw [dashed, very thick,->] (cnn) -- (lstm);
        \draw [dashed, very thick,->] (lstm) -- (dropout);
        \draw [arrow] (dropout) -- (softmax);
        \draw [arrow] (softmax) -- (output);
        \draw [arrow] (output) -- (loss);
        \draw [arrow] (output) -- (akurasi);
        \draw [arrow] (target) -- (loss);
        \draw [arrow] (target) -- (akurasi);
        \draw [arrow] (loss) -- (optimizer);
        \draw [arrow] (optimizer) -- (cnn);
        \draw [arrow] (optimizer) -- (lstm);
        \draw [arrow] (optimizer) -- (dropout);
        \draw [arrow] (optimizer) -- (softmax);
    \end{tikzpicture}
    \caption{Aliran data pada proses klasifikasi dan pelatihan}
    \label{gambar:aliran-data-klasifikasi}
\end{figure}

\section{Model Klasifikasi}

Proses klasifikasi dilakukan dalam beberapa langkah. Pertama, pra pengolahan dilakukan pada data masukan untuk pengondisian. Data tersebut kemudian dimasukkan ke dalam beberapa lapisan konvolusi. Setelah itu hasil lapisan konvolusi dimasukkan ke dalam beberapa lapisan LSTM\@. Kedua lapisan ini menghasilkan fitur-fitur abstrak yang merepresentasikan data sensor. Fitur-fitur ini kemudian diklasifikasikan dalam lapisan \textit{softmax}. Lapisan \textit{softmax} menghasilkan keluaran distribusi peluang kelas aktivitas dari data masukan. Hubungan antara lapisan LSTM dan lapisan \textit{sotfmax} diregulasi dengan \textit{dropout} untuk mengurangi \textit{overfitting}.

\subsection{Data Masukan}
Data masukan adalah rangkaian data \textit{time series} dari sensor akselerometer dan giroskop. Rangkaian data tersebut diambil dari ponsel cerdas dan dikelompokkan menggunakan \textit{sliding window} pada setiap 100 langkah waktu dengan \textit{overlap} 50\%.

Sensor akselerometer dan giroskop menghasilkan nilai bacaan dengan tiga sumbu, yaitu sumbu $x$, $y$ dan $z$. Nilai sensor akselerometer $\pmb{a}$ dan giroskop $\pmb{g}$ dinotasikan sebagai vektor~\ref{eq:vektor-akselerometer} dan~\ref{eq:vektor-giroskop}.

\begin{equation}
    \label{eq:vektor-akselerometer}
    \pmb{a} = 
    \begin{bmatrix}
        a_x \\
        a_y \\
        a_z
    \end{bmatrix}
\end{equation}

\begin{equation}
    \label{eq:vektor-giroskop}
    \pmb{g} = 
    \begin{bmatrix}
        g_x \\
        g_y \\
        g_z
    \end{bmatrix}
\end{equation}

Nilai sensor pada sumbu $x$, $y$ dan $z$ dipengaruhi oleh posisi dan arah gerak sensor. Mengingat posisi ponsel cerdas saat digunakan tidak menentu, maka perlu dicari nilai yang tidak dipengaruhi oleh arah gerak sensor, yaitu besar dari masing-masing vektor $\pmb{a}$ dan $\pmb{g}$. Besar vektor akseleromter dan giroskop dapat dicari dengan Persamaan~\ref{eq:norm-akselerometer} dan~\ref{eq:norm-giroskop}.

\begin{equation}
    \label{eq:norm-akselerometer}
    |\pmb{a}| = \sqrt{a_x^2 + a_y^2 + a_z^2}
\end{equation}

\begin{equation}
    \label{eq:norm-giroskop}
    |\pmb{g}| = \sqrt{g_x^2 + g_y^2 + g_z^2}
\end{equation}

Model klasifikasi menerima masukan matriks $\pmb{X}_{t \times n}$ dengan ukuran $t = 100$ dan $n = 8$. Setiap baris tersebut merupakan \textit{sample} sensor akselerometer dan giroskop dalam satu langkah waktu yang diurutkan dari waktu $t_1$ sampai $t_{100}$. Matriks $\pmb{X}$ ditunjukkan dalam Persamaan~\ref{eq:matriks-masukan}.

\begin{equation}
    \label{eq:matriks-masukan}
    \pmb{X}_{t \times n} =
    \begin{bmatrix}
        a_x^{(0)} & a_y^{(0)} & a_z^{(0)} & |\pmb{a}|^{(0)} & g_x^{(0)} & g_y^{(0)} & g_z^{(0)} & |\pmb{g}|^{(0)} \\
        a_x^{(1)} & a_y^{(1)} & a_z^{(1)} & |\pmb{a}|^{(1)} & g_x^{(1)} & g_y^{(1)} & g_z^{(1)} & |\pmb{g}|^{(1)} \\
        \cdots & \cdots & \cdots & \cdots & \cdots & \cdots & \cdots & \cdots \\
        a_x^{(100)} & a_y^{(100)} & a_z^{(100)} & |\pmb{a}|^{(100)} & g_x^{(100)} & g_y^{(100)} & g_z^{(100)} & |\pmb{g}|^{(100)} \\
        
    \end{bmatrix}
\end{equation}

\subsection{Lapisan Konvolusi}
Diagram blok lapisan konvolusi ditunjukkan pada Gambar~\ref{gambar:diagram-blok-kovolusi}. Matriks masukan $\pmb{X}$ dilewatkan melalui lapisan kovolusi untuk mencari fitur-fitur dari data sensor. Lapisan konvolusi menggunakan kernel $\pmb{K}_{m \times n}$, masing-masing nilai anggotanya adalah bobot yang digunakan bersama oleh setiap jendela komputasi. Jumlah lapisan konvolusi dan ukuran kernel $m \times n$ merupakan \textit{hyperparameter} yang diatur sebelum pelatihan.

\begin{figure}[h!]
    \centering
    \begin{tikzpicture}[node distance=1.5cm]
        \node (input) [process] {Input};
        \node (konvolusi) [process, above of=input, yshift=0.5cm] {Konvolusi};
        \node (relu) [process, above of=konvolusi] {Rectified Linear Unit};
        \node (output) [process, above of=relu, yshift=1.2cm] {Output};
        \draw (-2.5,4.8) -- node[anchor=north] {Lapisan Konvolusi} (2.5,4.8) -- (2.5,1) -- (-2.5,1) -- (-2.5,4.8);

        \draw [arrow] (input) -- (konvolusi);
        \draw [arrow] (konvolusi) -- (relu);
        \draw [arrow] (0,4.8) -- (output);
    \end{tikzpicture}
    \caption{Diagram blok lapisan konvolusi}
    \label{gambar:diagram-blok-kovolusi}
\end{figure}

Nilai keluaran lapisan konvolusi dapat dihitung dengan melakukan konvolusi antara jendela komputasi $\pmb{I}$ dengan bobot $\pmb{W}$. Hasil konvolusi ditambahkan dengan bias dan dimasukkan ke dalam fungsi aktivasi \textit{Rectified Linear Unit} (ReLU)\@. Operasi tersebut didefinisikan sebagai

\begin{equation}
    H(i,j) = g((\pmb{I} * \pmb{W})(i,j) + \pmb{b})
    \label{eq:konvolusi-kernel}
\end{equation}

\noindent
dengan

\begin{equation}
    (\pmb{I} * \pmb{W})(i,j) = \sum_{m}\sum_{n}I(m,n)W(i-m, j-n)
    \label{eq:konvolusi-3d}
\end{equation}

\noindent
dan fungsi aktivasi ReLU didefinisikan dalam persamaan

\begin{equation}
    g(z) = \max(0,z)
    \label{eq:relu}
\end{equation}

Perhitungan dilakukan untuk setiap jendela komputasi. Berikut ini tahapan perhitungan dalam lapisan konvolusi:

\begin{enumerate}
\item Jendela komputasi $\pmb{I}$ dibuat dari $\pmb{X}$ dengan ukuran $m \times n$
\item Untuk setiap jendela, lakukan perhitungan konvolusi dengan Persamaan~\ref{eq:konvolusi-kernel}.
\item Lakukan tahap 1 dan 2 dengan langkah antar jendela 1
\item Susun hasil perhitungan setiap jendela menjadi matriks $\pmb{H}$
\end{enumerate}

\subsection{Lapisan LSTM}
Output dari lapisan konvolusi dilewatkan melalui lapisan LSTM\@. Jumlah lapisan LSTM $L$ dan jumlah \textit{hidden unit} pada masing-masing lapisan merupakan \textit{hyperparameter} yang ditentukan sebelum pelatihan. Setiap lapisan memiliki bobot antar input ke hidden $\pmb{U}$, bobot antar hidden ke output $\pmb{V}$, bobot antar hidden ke hidden $\pmb{W}$, serta bias $\pmb{b}$ dan $\pmb{c}$. Jika diberikan input matriks $\pmb{X}_{m \times n}$, $\pmb{x}^{(t)}$ merupakan subset dari $\pmb{X}$ pada langkah waktu $t$ dari $t = 1$ sampai $t = m$. Keluaran LSTM $\pmb{h}^{(t)}$ dapat dicari dengan Persamaan~\ref{eq:output-lstm}.

\begin{figure}[h!]
    \centering
    \begin{tikzpicture}[node distance=2cm]
        \node (x1) [cell] {$x^{(1)}$};
        \node (x2) [cell, right of=x1] {$x^{(2)}$};
        \node (xm) [cell, right of=x2, xshift=1cm] {$x^{(m)}$};
        \node (h1) [cell, above of=x1] {$h^{(1)}$};
        \node (h2) [cell, above of=x2] {$h^{(2)}$};
        \node (hm) [cell, above of=xm] {$h^{(m)}$};
        \node (y1) [cell, above of=h1] {$y^{(1)}$};
        \node (y2) [cell, above of=h2] {$y^{(2)}$};
        \node (ym) [cell, above of=hm] {$y^{(m)}$};

        \draw [arrow] (x1) -- (h1);
        \draw [arrow] (x2) -- (h2);
        \draw [arrow] (xm) -- (hm);
        \draw [arrow] (h1) -- (y1);
        \draw [arrow] (h2) -- (y2);
        \draw [arrow] (hm) -- (ym);
        \draw [arrow] (h1) -- (h2);
        \draw [dashed, very thick,->] (h2) -- (hm);
    \end{tikzpicture}
    \caption{Arsitektur lapisan LSTM \textit{many-to-many}}
    \label{gambar:many-to-many}
\end{figure}

\begin{figure}[h!]
    \centering
    \begin{tikzpicture}[node distance=2cm]
        \node (x1) [cell] {$x^{(1)}$};
        \node (x2) [cell, right of=x1] {$x^{(2)}$};
        \node (xm) [cell, right of=x2, xshift=1cm] {$x^{(m)}$};
        \node (h1) [cell, above of=x1] {$h^{(1)}$};
        \node (h2) [cell, above of=x2] {$h^{(2)}$};
        \node (hm) [cell, above of=xm] {$h^{(m)}$};
        \node (y) [cell, above of=hm] {$y$};

        \draw [arrow] (x1) -- (h1);
        \draw [arrow] (x2) -- (h2);
        \draw [arrow] (xm) -- (hm);
        \draw [arrow] (hm) -- (y);
        \draw [arrow] (h1) -- (h2);
        \draw [dashed, very thick,->] (h2) -- (hm);
    \end{tikzpicture}
    \caption{Arsitektur lapisan LSTM \textit{many-to-one}}
    \label{gambar:many-to-one}
\end{figure}

Sel-sel LSTM pada lapisan pertama sampai lapisan ke $L-1$ disusun dengan arsitektur \textit{many-to-many} seperti pada Gambar~\ref{gambar:many-to-many}, sedangkan sel-sel LSTM pada lapisan ke-$L$ disusun dengan arsitektur \textit{many-to-one} seperti pada Gambar~\ref{gambar:many-to-one}. Input lapisan LSTM pertama diambil dari output lapisan konvolusi, sedangkan input lapisan LSTM selanjutnya diambil dari output lapisan LSTM sebelumnya. Output dari lapisan LSTM terakhir hanya diambil dari keluaran pada langkah waktu terakhir, yaitu $\pmb{h}^{(t)}$ pada saat $t = m$.

\subsection{Dropout dan Lapisan Softmax}
Output dari lapisan LSTM masuk ke lapisan \textit{softmax} dan diregulasi dengan Dropout untuk mencegah \textit{overfitting}. Lapisan \textit{softmax} adalah jaringan padat yang menggunakan fungsi \textit{softmax} sebagai fungsi aktivasinya. Fungsi aktivasi \textit{softmax} digunakan untuk mencari distribusi peluang seluruh kelas aktivitas manusia.

Jaringan padat adalah jaringan dengan neuron yang seluruhnya terhubung dengan neuron pada lapisan sebelum atau sesudahnya. Jaringan padat dengan enam neuron dibuat dengan masukan dari hasil lapisan LSTM terakhir. Jumlah neuron tersebut dipilih sesuai dengan jumlah aktivitas yang akan diklasifikasikan.

Lapisan \textit{softmax} menerima masukan vektor $\pmb{x}$. Dengan bobot $\pmb{W}$ dan bias $\pmb{b}$, keluaran lapisan ini dihitung menggunakan persamaan

\begin{equation}
    \pmb{y} = softmax(\pmb{W} \pmb{x} + \pmb{b})
\end{equation}

\noindent
dan fungsi \textit{softmax} didefinisikan sebagai

\begin{equation}
    softmax(\pmb{z})_i = \frac{\exp(z_i)}{\sum_{j=1}^n \exp(z_j)}
\end{equation}

Persamaan di atas menghasilkan vektor distribusi peluang dari enam kelas aktivitas. Aktivitas yang dikenali adalah aktivitas dengan peluang tertinggi.


\section{Pelatihan}
Proses pelatihan jaringan dilakukan untuk mencari bobot dan bias yang optimal pada seluruh neuron dalam setiap lapisan. Optimasi bobot dan bias tersebut dilakukan dengan meminimal \textit{cost function} menggunakan algoritma optimasi berbasis \textit{gradient descent}. \textit{Cost function} yang digunakan adalah \textit{cross entropy} dan algoritma \textit{gradient descent} yang digunakan adalah RMSProp \Parencite{Dauphin-2015}.

% \subsection{Perhitungan \textit{loss} dengan Cross Entropy}

% \subsection{Optimasi dengan RMSProp}

% \section{Pemilihan Hyperparameter dengan Pencarian Acak}

\section{Data Pelatihan}
Model klasifikasi dilatih dengan dataset yang dikumpulkan dari 48 parisipan (36 laki-laki dan 12 perempuan) dalam rentang usia antara 20 sampai 28 tahun. Setiap partisipan melakukan aktivitas duduk, berdiri, berjalan, berlari, menaiki tangga dan menuruni tangga dengan menempatkan ponsel cerdas di saku celana. Data sensor akselerometer dan giroskop diambil dengan \textit{sampling rate} 50 Hz, lalu \textit{sliding window} dibuat dari setiap 100 sample dengan \textit{overlap} 50\%. Dengan aturan tersebut, diperoleh data aktivitas seperti pada Tabel~\ref{table:jumlah-dataset}.

\begin{table}[h!]
    \centering
    \caption{Jumlah dataset}
    \begin{tabular}{ |l|c| }
        \hline
        Aktivitas & Jumlah Sampel \\

        \hline
        Berdiri & 14.394 \\

        \hline
        Duduk & 1.710 \\

        \hline
        Berjalan & 14.394 \\

        \hline
        Berlari & 4.314 \\

        \hline
        Menaiki Tangga & 2.844 \\

        \hline
        Menuruni Tangga & 2.864 \\

        \hline
        Total & 40.520 \\

        \hline
    \end{tabular}
    \label{table:jumlah-dataset}
\end{table}

Proses pelatihan membutuhkan tiga jenis data, yaitu data latih, data validasi dan data uji. Data latih digunakan untuk melatih model dalam optimasi parameter yang menghasilkan klasifikasi terbaik, data validasi digunakan untuk mengukur kemampuan model seiring berjalannya proses palatihan, sedangkan data uji digunakan untuk mengukur kemampuan model akhir hasil pelatihan. Data latih, data validasi dan data uji diambil dari partisipan yang berbeda. Pembagian data tersebut dapat dilihat pada Tabel~\ref{table:pembagian-data}.

\begin{table}[h!]
    \centering
    \caption{Pembagian data latih, validasi dan uji}
    \begin{tabular}{ |l|c|c| }
        \hline
        Jenis Data & Partisipan & Jumlah Sampel \\

        \hline
        Data Latih & 27 & 22.776 \\

        \hline
        Data Validasi & 7 & 5.928 \\

        \hline
        Data Uji & 14 & 11.816 \\

        \hline
        Total & 48 & 40.520 \\

        \hline
    \end{tabular}
    \label{table:pembagian-data}
\end{table}

\section{Klasifikasi Pada Ponsel Cerdas}
Proses klasifikasi pada ponsel cerdas ditunjukkan pada Gambar~\ref{gambar:diagram-alir-klasifikasi-ponsel-cerdas}. Sensor akselerometer dan giroskop dibaca dengan \textit{sampling rate} 50 Hz. Masing-masing sensor menghasilkan data dengan tiga sumbu, sehingga satu kali pembacaan menghasilkan enam data. Hasil bacaannya dikumpulkan menjadi jendela berisi 100 sample dengan tumpang tindih 50\%. Dengan begitu, satu jendela data terdiri dari pembacaan sensor selama dua detik dan jendela baru dibuat setiap satu detik. Setelah jendela terpenuhi, jendela tersebut dijadikan sebagai masukan model klasifikasi yang telah dibuat. Proses ini menghasilkan kelas aktivitas manusia yang sesuai dengan bacaan sensor.

\begin{figure}[h]
    \centering
    \begin{tikzpicture}[node distance=2cm]
        \node (mulai) [startstop] {Mulai};
        \node (baca-sensor) [io, below of=mulai] {Baca sensor};
        \node (susun-jendela) [process, below of=baca-sensor] {Susun jendela};
        \node (jendela-siap) [decision, below of=susun-jendela, yshift=-1cm] {\footnotesize{Ukuran jendela == 100}};
        \node (klasifikasi) [process, below of=jendela-siap, yshift=-1cm] {Klasifikasi};
        \node (selesai) [startstop, below of=klasifikasi] {Selesai};

        \draw [arrow] (mulai) -- (baca-sensor);
        \draw [arrow] (baca-sensor) -- (susun-jendela);
        \draw [arrow] (susun-jendela) -- (jendela-siap);
        \draw [arrow] (jendela-siap) -- node[anchor=south] {tidak} (4,-7) -- (4,-2) -- (baca-sensor);
        \draw [arrow] (jendela-siap) -- node[anchor=west] {ya} (klasifikasi);
        \draw [arrow] (klasifikasi) -- (selesai);
    \end{tikzpicture}
    \caption{Diagram alir klasifikasi pada ponsel cerdas}
    \label{gambar:diagram-alir-klasifikasi-ponsel-cerdas}
\end{figure}

%------------------------------------------------------------------------------
% Pengujian
%
% Tabel: indikator keberhasilan
% Menjawab judul rumusan masalah
%------------------------------------------------------------------------------
\section{Rencana Pengujian}
Pengujian dilakukan untuk mengetahui kemampuan klasifikasi dari model yang telah dibuat. Terdapat dua pengujian yang dilakukan, yaitu pengujian \textit{hyperparameter} dan pengujian kecepatan klasifikasi pada ponsel cerdas. Kedua pengujian ini dilakukan dengan indikator pencapaian yang ditunjukkan pada Tabel~\ref{table:rencana-pengujian}.

Pada pengujian \textit{hyperparameter}, berbagai konfigurasi model diuji untuk mencari hyperparameter terbaik. Kemampuan model klasifikasi diukur berdasarkan akurasi klasifikasi pada data uji. Model klasifikasi menerima data uji yang diambil secara acak dari dataset dan terpisah dari data latih. Output dari proses klasifikasi dibandingkan dengan label data tersebut.

Pengujian kecepatan klasifikasi pada ponsel cerdas dilakukan dengan mengukur waktu yang dibutuhkan oleh setiap siklus klasifikasi, dimulai dari selesainya pengambilan data sensor sampai diperoleh kelas aktivitas dari data tersebut. Pengukuran ini dilakukan dengan beberapa model ponsel cerdas dengan kemampuan pemrosesan yang berbeda.

\begin{table}[ht]
    \centering
    \caption{Rencana Pengujian}
    \begin{tabular}{ |p{0.5cm}|p{5cm}|p{7.5cm}| }
        \hline
        \textbf{No} & \textbf{Parameter} & \textbf{Pencapaian} \\

        \hline
        1 & Pengujian Hyperparameter & Didapatkan hyperparameter yang menghasilkan klasifikasi terbaik pada data uji \\
        
        \hline
        2 & Pengujian kecepatan klasifikasi pada ponsel cerdas & Klasifikasi dapat dilakukan secara \textit{realtime} pada ponsel cerdas \\

        \hline
    \end{tabular}
    \label{table:rencana-pengujian}
\end{table}
