\preface
Puji syukur penulis ucapkan kepada Allah SWT atas kuasa-Nya yang telah melimpahkan semua kasih sayang, rahmat, hidayah dan karunia-Nya sehingga penulis dapat menyelesaikan skripsi yang berjudul "Pengenalan Aktivitas Manusia Menggunakan Sensor pada Smartphone dengan \textit{Convolutional Neural Network} dan \textit{Long Short-Term Memory}".

Skripsi ini disusun sebagai salah satu syarat untuk memperoleh gelar Sarjana S1 Program Studi Elektronika dan Instrumentasi, Departemen Ilmu Komputer dan Elektronika, Fakultas Matematika dan Ilmu Pengetahuan Alam, Universitas Gadjah Mada Yogyakarta.

Penulis menyadari bahwa selesaiannya skripsi ini tidak terlepas dari bantuan dan dukungan dari berbagai pihak. Oleh karena itu, dengan segala hormat pada kesempatan ini penulis ingin menyampaikan penghargaan dan terima kasih yang sebesar-besarnya kepada:

\begin{enumerate}
    \item Kedua pasangan orang tua, kakak, dan adik yang selalu memberikan doa, semangat, nasehat, dorongan dan bantuan kepada penulis selama ini.
    \item Bapak R. Sumiharto, S.Si. M.Kom. selaku dosen pembimbing skripsi yang telah banyak meluangkan waktu untuk membimbing, memberikan ide dan pemikiran, serta saran dan masukan sehingga penulis dapat menyelesaikan skripsi ini dengan baik.
    \item Dosen-dosen penulis selama mengikuti perkuliahan di program studi Elektronika dan Instrumentasi ini yang telah memberikan banyak ilmu dan bimbingan.
    \item Semua partisipan yang telah membantu dalam pengambilan data yang tidak bisa disebutkan satu per satu namanya.
    \item Teman-teman seperjuangan di komunitas N2 yang telah memberikan inspirasi, motivasi, semangat dan masukan (Rifa, Oci, Mas Arif, Mas Dito, Mas Aries, Mas Dimas, Mas Dien, Mas Ade, Mas Aga, dan kakak-kakak senior yang tidak bisa disebutkan satu-satu namanya).
    \item Teman-teman seperjuangan di KSK \textit{Electronics Research Laboratory} (Ahmad, Ibnu, Adit, Ridho, Ilham) yang selalu memberikan semangat dan membantu ketika ada kesulitan.
    \item Teman-teman KKN-PPN Pulo Aceh yang telah memberikan motivasi, pembelajaran dan pengalaman hidup yang tak terlupakan.
    \item Semua teman-teman Elins angkatan 2013 atas dukungan, kebersamaan dan kesediannya berbagi ilmu.
    \item Nindya Rachma Latifani yang telah menemani dalam masa-masa senang dan sulit, serta memotivasi penulis untuk menyelesaikan penelitian ini dengan baik.
    \item Semua pihak yang selama ini telah membantu, mendukung dan menyemangati penulis hingga saat ini.
\end{enumerate}

Semoga Allah selalu memberikan rahmat serta kemudahan kepada semua pihak yang telah banyak membantu dalam penyelesaiak skripsi ini. Peneliti menyadari bahwa skripsi ini tidak sempurna, oleh karena itu saran dan kritik yang positif sangat diharapkan. Semoga skripsi ini dapat berguna pagi peneliti dan pembaca.

\vspace{1.5cm}
\begin{tabular}{p{7.5cm}c}
&Yogyakarta, 30 Agustus 2017\\
&\\
&\\
&Penulis
\end{tabular}
\vfill